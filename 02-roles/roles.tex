\documentclass[usenames,dvipsnames]{beamer}
    \mode<presentation> {
    \usetheme{Montpellier}
    \usecolortheme{beaver}
    %\setbeamertemplate{footline} % To remove the footer line in all slides uncomment this line
    \setbeamertemplate{footline}[page number] % To replace the footer line in all slides with a simple slide count uncomment this line

    \setbeamertemplate{navigation symbols}{} % To remove the navigation symbols from the bottom of all slides uncomment this line
    }

    \usepackage{graphicx} % Allows including images
    \usepackage{booktabs} % Allows the use of \toprule, \midrule and \bottomrule in tables
    \usepackage{minted}
    \usepackage{xcolor}
    %\usepackage[mathletters]{ucs}
    %\usepackage[utf8x]{inputenc}
    \usepackage[utf8]{inputenc}
    \usepackage{pifont}    
    \usepackage{newunicodechar}
    \newunicodechar{✪}{\ding{74}}

    %\DeclareUnicodeCharacter{272A}{\star}
    \definecolor{mintedbackground}{rgb}{0.95,0.95,0.95}
    \newcommand{\code}[1]{\colorbox{lightgray}{\texttt{#1}}}


\newmintedfile[scalacode]{scala}{
bgcolor=mintedbackground,
fontfamily=tt,
linenos=true,
numberblanklines=true,
numbersep=5pt,
gobble=0,
frame=leftline,
framerule=0.4pt,
framesep=2mm,
funcnamehighlighting=true,
tabsize=4,
obeytabs=false,
mathescape=false
samepage=false, %with this setting you can force the list to appear on the same page
showspaces=false,
showtabs =false,
texcl=false,
}
    \setminted{fontsize=\footnotesize,baselinestretch=1}

    \usepackage {tikz}
    \usetikzlibrary {positioning}
    \graphicspath {{target/media/}}
    %----------------------------------------------------------------------------------------
    %	TITLE PAGE
    %----------------------------------------------------------------------------------------

    \title[Roles]{Roles: a viable alternative to Microservices and Monoliths}

    \institute[Septimal Mind Ltd]
    {
    Septimal Mind Ltd\\
    \medskip
    \textit{team@7mind.io}
    }
    \date{\today}

\begin{document}

\begin{frame}
\titlepage
\end{frame}

\section{The problem: both Monolith and Microservice approaches are broken}

\begin{frame}
\frametitle{Monoliths are broken}
We all know that Monoliths aren't nice:
\begin{itemize}
\item Inconvenient when your codebase is large: you may need to rebuild to much, etc,
\item Provoking tight-coupling and non-SOLID code,
\item Startup time is usually high,
\item When you update you need to update everything (maybe not a bad things though),
\item Issues with ,
\item Some issues may be kinda mitigated by taking care of artifact structuring,
      pulling conservative parts out of the main codebase into independent libraries.
\end{itemize}
\end{frame}

\begin{frame}
\frametitle{You've seen this before}
\begin{figure}
    \includegraphics[width=0.9\textwidth]{media/mono-ms-joke.jpg}
\end{figure}
\end{frame}

\begin{frame}
\frametitle{The alternative -- Microservices -- is just sick}

\begin{itemize}
\item Evil Dependencies and neccessary \textit{Shared State},
\item Deployment and maintenance becomes more complicated with each new dependency,
\item Inevitably Distributed: all \textit{fundamental} problems are in place. CAP theorem?,
\item Cascade failures,
\item Accidental Incompatibilities: you may notice them too late. Amplified by cascade failures,
\item Computing Density: 50 JVM microservices in Docker on a single machine? Really?,
\item Overcomplicated Development Flows: simple (integration\footnotemark[1]) test requires many things done in right order,
\item Time, interference and reproducible builds: isolated environments are expensive to setup.
\end{itemize}
\footnotetext[1]{Not a constructive definition. Subject of another presentation}
\end{frame}

\begin{frame}
\frametitle{Observations on Microservices}
\begin{itemize}
\item \textit{Software components} are still in place -- in form of Services.
      All the potential pitfails of component design apply to microservices,
\item Many checks which may be done by your compiler in case of monolithic design have  
     to be performed by integration test suites (in case you have them, it's hard to write),   
\item Many tasks are delegated to operations: in case of a monolith your dependency graph is being 
      processed by your DI framework, in case of microservices it is\dots in worst case processed by people,
      in best case requires an orchestration tool. But we don't have any single great tool yet,
\item An orchestration tool does \textit{exactly the same job} your DI framework may do. And lot more things.
\end{itemize}
\end{frame}

\section{The solution: roles}

\begin{frame}
\frametitle{Roles: the idea to rescue}
What we need?
\begin{itemize}
\item Something with all the positive traits of Microservices and Monoliths but without negative.
\end{itemize}
\vspace{0.3cm}
So:
\begin{itemize}
\item Let's run multiple services\dots in one service. Well, one process,
\item Let's choose which \textit{roles} we wish to assign to a  before we start it,
\item It resembles Containers and OSGi,
\item Containers have some disadvantages. 
      They provide you a lot so we get used to think that they are sluggish on startup.
      Isolated classloaders are inconvenient. Dynamic DI is a nightmare,
\item Let's get rid of Dynamic DI. And let's have isolation optional.
\end{itemize}
\end{frame}

\begin{frame}
\frametitle{Benefits}
\begin{itemize}
\item 
\end{itemize}
\end{frame}

\begin{frame}
\frametitle{Missing things and potential pitfails}
\begin{itemize}
\item Startup time. Do you remember Tomcat? You have to pay attention to your design -- and there are some recepies\footnotemark[1],
\item Heterogenous systems: you can't fit C\#, Scala and Go into a single process. 
      Though you still can greatly improve heterogenous systems by using roles,
\item Dependency convergence. In case you want roles to be completely independent you need
      isolated classloaders. Solved in OSGi. Expensive, inconvenient.
\end{itemize}
\footnotetext[1]{We have something to add to well-known answers. Subject of another presentation}
\end{frame}

\section{The implementation: distage}

\begin{frame}
\begin{figure}
\Huge 
\color{RubineRed} DISTAGE
\noindent
{\color{RubineRed} \rule{\linewidth}{1mm} }
\Large Next-gen Dependency Injection Framework for Scala
\normalsize Generative, Modular, R✪les and Garbage Collection included 
\end{figure}
\end{frame}

\begin{frame}
\frametitle{Quick overview}
\begin{itemize}
\item 
\end{itemize}
\end{frame}

\begin{frame}
\frametitle{Status and things to do}
Distage is:
\begin{itemize}
\item ready to use,
\item in real production for 4 months.
\end{itemize}
\vspace{0.3cm}
Our plans:
\begin{itemize}
\item Make Roles opensource. ASAP,
\item Support optional isolated classloaders (in foreseeable future),
\item Check our GitHub: https://github.com/pshirshov/izumi-r2.
\end{itemize}
\end{frame}

\begin{frame}
\frametitle{DIStage is just a part of our stack}
We have a vision backed by our tools:
\begin{itemize}
\item Idealingua: transport and codec agnostic GRPC alternative with rich modeling language,
\item LogStage: zero-cost logging framework,
\item \textit{Fusional Programming and Design} guidelines. We love both FP and OOP,
\item \textit{Continous Delivery} guidelines for Role-based process, 
\item \textit{Percept-Plan-Execute} Generative Programming approach, abstract machine and computational model.
Addresses Project Planning (see Operations Research). Examples: orchestration, build systems.
\end{itemize}

Altogether these things already allowed us to significantly reduce development costs and
delivery time for our client.\newline

More slides to follow.
\end{frame}

\begin{frame}
    \frametitle{Thank you for attention}

    \begin{center}
      https://izumi.7mind.io/distage/

      We're looking for clients, contributors, adopters and colleagues ;)
    \end{center}

    About the author:
    \begin{itemize}
        \item coding for 18 years, 10 years of hands-on commercial engineering experience,
        \item has been leading a cluster orchestration team in Yandex, ``the Russian Google'',
        \item implemented ``\textit{Interstellar Spaceship}'' -- an orchestration solution to manage 50K+ physical machines across 6 datacenters,
        \item Owns an Irish R\&D company, https://7mind.io,
        \item Contacts: team@7mind.io,
        \item Github: https://github.com/pshirshov
    \end{itemize}
\end{frame}

\end{document}
