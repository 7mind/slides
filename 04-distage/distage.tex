\documentclass[usenames,dvipsnames]{beamer}
    \mode<presentation> {
    \usetheme{Montpellier}
    \usecolortheme{beaver}
    %\setbeamertemplate{footline} % To remove the footer line in all slides uncomment this line
    \setbeamertemplate{footline}[page number] % To replace the footer line in all slides with a simple slide count uncomment this line
    \setbeamertemplate{navigation symbols}{} % To remove the navigation symbols from the bottom of all slides uncomment this line
    }

    \usepackage{graphicx} % Allows including images
    \usepackage{booktabs} % Allows the use of \toprule, \midrule and \bottomrule in tables
    \usepackage{minted}
    \usepackage{xcolor}
    \usepackage[utf8]{inputenc}
    \usepackage{pifont}    
    \usepackage{newunicodechar}
    \newunicodechar{✪}{\ding{74}}

    \definecolor{mintedbackground}{rgb}{0.95,0.95,0.95}
    \newcommand{\code}[1]{\colorbox{lightgray}{\texttt{#1}}}


\newmintedfile[scalacode]{scala}{
bgcolor=mintedbackground,
fontfamily=tt,
linenos=true,
numberblanklines=true,
numbersep=5pt,
gobble=0,
frame=leftline,
framerule=0.4pt,
framesep=2mm,
funcnamehighlighting=true,
tabsize=4,
obeytabs=false,
mathescape=false
samepage=false, %with this setting you can force the list to appear on the same page
showspaces=false,
showtabs =false,
texcl=false,
}
    \setminted{fontsize=\footnotesize,baselinestretch=1}

    \usepackage {tikz}
    \usetikzlibrary {positioning}
    \graphicspath {{target/media/}}

    \title[DIStage]{DIStage: Modern Staged Dependency Injection for Scala}

    \institute[Septimal Mind Ltd]
    {
    Septimal Mind Ltd\\
    \medskip
    \textit{team@7mind.io}
    }
    \date{\today}

\begin{document}

% \begin{VerbatimOut}{ex-scala-roles.tmp}
% @RoleId("testservice") 
% class TestService[F[_] : Monad](http: HttpSrv[F]) 
%   extends IzService {
%     override def start(): Unit = http.start()
%     override def stop(): Unit = http.stop()
% }
% class TestPlugin extends PluginDef {
%   many[IzService].add[TestService[IO]]
% }
% object TestLauncher {
%   // you may run it like `test.jar test-service other-service`
%   def main(args: Array[String]): Unit = IzRoleApp(args).main()
% }
% \end{VerbatimOut}

\begin{frame}
%\titlepage
\begin{figure}
\Huge 
\color{RubineRed} DIST✪GE
\noindent
\rule{\linewidth}{1mm}
\Large Modern Staged Dependency Injection for Scala
\rule{\linewidth}{1mm}
\end{figure}

\begin{figure}
\color{RubineRed}
\normalsize Type\-level Modular Code \\
with \\
Context Minimization \\
through \\
Garbage Collection
\end{figure}

\begin{figure}
\Large Septimal Mind Ltd \\
\medskip
\textit{team@7mind.io}
\end{figure}

\end{frame}

% - Modularity and it's importance 
% - DI-like mechanisms and their issues, especially in context of Scala (runtime DI is unreliable and does not integrate well with typesystem, compile-time DI may be not flexible enough) 
% - Why people get used to think that "DI does not compose with functional programming" why is it not correct and why alternatives are not good enough 
% - How we designed a staged DI which may work in both runtime and compile time 
% - How we may we improve reliability and get additional guarantees by exploiting staged approach 
% - Our integration with Scala typesystem 
% - how we support type tags for type lambdas and how we deal with implicits in runtime - and how may we write modular type-level code (so, we may use DI together with tagless final style, with IO, etc) 
% - Garbage-collecting DI and how may it help to significantly improve tests and deployments ( related slides: https://github.com/7mind/slides/blob/master/02-roles/target/roles.pdf )

\section{The problem: Dependency Injection and Functional Programming}

\begin{frame}
\frametitle{The motivation behind DI}
\begin{itemize}
\item Systems we work with may be represented as graphs of interlinked components where nodes are components (usually instances) and edges are references,
\item The graph 
\end{itemize}
\end{frame}

\begin{frame}
\frametitle{``DI doesn't compose with FP''}
\begin{itemize}
\item 
\end{itemize}
\end{frame}

\section{DIStage: Staged DI for Scala}

\begin{frame}
\frametitle{DI implementations are broken}
So we may build better one:
\begin{itemize}
\item 
\end{itemize}
\end{frame}

\begin{frame}
\frametitle{Staged approach}
\begin{itemize}
\item 
\end{itemize}
\end{frame}

\begin{frame}
\frametitle{Compile-Time and Runtime DI}
\begin{itemize}
\item 
\end{itemize}
\end{frame}

\begin{frame}
\frametitle{Extension: Configuration Support}
\begin{itemize}
\item 
\end{itemize}
\end{frame}

\begin{frame}
\frametitle{Pattern: Plan Completion}
\begin{itemize}
\item 
\end{itemize}
\end{frame}

\begin{frame}
\frametitle{The Principle Behind: PPER}
\begin{itemize}
\item 
\end{itemize}
\end{frame}

\subsection{Garbage Collector and its Benefits}

\begin{frame}
\frametitle{Garbage Collector}
\begin{itemize}
\item 
\end{itemize}
\end{frame}

\begin{frame}
\frametitle{Context Minimization}
\begin{itemize}
\item 
\end{itemize}
\end{frame}

\begin{frame}
\frametitle{Context Minimization for Tests}
\begin{itemize}
\item 
\end{itemize}
\end{frame}

\begin{frame}
\frametitle{Context Minimization for Deployment}
\begin{itemize}
\item 
\end{itemize}
\end{frame}

\subsection{Scala Typesystem Integration: Fusional Programming}

\begin{frame}
\frametitle{Kind-Polymorphic Type Tags}
\begin{itemize}
\item 
\end{itemize}
\end{frame}

\begin{frame}
\frametitle{Typeclass instance injection (Implicit Injection)}
\begin{itemize}
\item 
\end{itemize}
\end{frame}

\begin{frame}
\frametitle{Code example: IO Injection}
\begin{itemize}
\item 
\end{itemize}
\end{frame}

\begin{frame}
\frametitle{Code example: Tagless Final Style}
\begin{itemize}
\item 
\end{itemize}
\end{frame}

\subsection{Convenience features}

\begin{frame}
\frametitle{Dynamic Plugins}
\begin{itemize}
\item 
\end{itemize}
\end{frame}

\begin{frame}
\frametitle{Tags}
\begin{itemize}
\item 
\end{itemize}
\end{frame}

\begin{frame}
\frametitle{Plan Introspection}
\begin{itemize}
\item 
\end{itemize}
\end{frame}

\begin{frame}
\frametitle{Trait Completion}
\begin{itemize}
\item Runtime and Compile-time.
\end{itemize}
\end{frame}

\begin{frame}
\frametitle{Factory Methods (Assisted Injection)}
\begin{itemize}
\item Useful for Akka, lot more convenient than Guice,
\item Runtime and Compile-time.
\end{itemize}
\end{frame}

\begin{frame}
\frametitle{Status and things to do}
Distage is:
\begin{itemize}
\item ready to use,
\item in real production,
\item all Runtime APIs are available,
\item Compile-time verification, trait completion, assisted injections and lambda injections are available.
\end{itemize}
\vspace{0.3cm}
Our plans:
\begin{itemize}
\item Refactor Roles API,
\item Support running Producer within a monad (IO),
\item Support Scala.js,
\item Support optional isolated classloaders (in foreseeable future),
\item Publish compile-time Producer,
\item Check our GitHub: https://github.com/pshirshov/izumi-r2.
\end{itemize}
\end{frame}

\begin{frame}
\frametitle{DIStage is just a part of our stack}
We have a vision backed by our tools:
\begin{itemize}
\item Idealingua: transport and codec agnostic gRPC alternative with rich modeling language,
\item LogStage: zero-cost logging framework,
\item \textit{Fusional Programming and Design} guidelines. We love both FP and OOP,
\item \textit{Continous Delivery} guidelines for Role-based process, 
\item \textit{Percept-Plan-Execute} Generative Programming approach, abstract machine and computational model.
Addresses Project Planning (see Operations Research). Examples: orchestration, build systems.
\end{itemize}

Altogether these things already allowed us to significantly reduce development costs and
delivery time for our client.\newline

More slides to follow.
\end{frame}

\begin{frame}
\frametitle{Teaser: LogStage}
\end{frame}

\begin{frame}
\frametitle{Teaser: Idealingua}
\end{frame}


\begin{frame}
    \frametitle{Thank you for your attention}

    \begin{center}
      https://izumi.7mind.io/

      We're looking for clients, contributors, adopters and colleagues ;)
    \end{center}

    About the author:
    \begin{itemize}
        \item coding for 18 years, 10 years of hands-on commercial engineering experience,
        \item has been leading a cluster orchestration team in Yandex, ``the Russian Google'',
        \item implemented ``\textit{Interstellar Spaceship}'' -- an orchestration solution to manage 50K+ physical machines across 6 datacenters,
        \item Owns an Irish R\&D company, https://7mind.io,
        \item Contacts: team@7mind.io,
        \item Github: https://github.com/pshirshov
    \end{itemize}
\end{frame}

\end{document}
